% !TEX encoding = UTF-8 Unicode
%!TEX root = main.tex
% !TEX spellcheck = en-US
%%=========================================

\chapter{Examples of Usage}
\label{ch:examples_usage}

This chapter shows the basics of ProXC and how to use this in a C program. Note that this is a very simple introduction, and only covers the basic.

\section*{Prerequisites}

To use ProXC in a C program, include the header file \texttt{proxc.h} in your C\hyp{}files, 

\begin{lstlisting}[style={CustomC},frame={},numbers={none}]
#include <proxc.h>
\end{lstlisting}

\noindent and during linking add the linker flag \texttt{-lproxc}. See Appendix \ref{ch:api_header} for the API header file \texttt{proxc.h}. From here on, ProXC can be used.

\section*{Starting and Exiting}

Everything starts with the \texttt{START} call. This is usually called from \texttt{main} function. A function pointer is supplied with the \texttt{START} call, which will start as the main process in the ProXC. When the main process returns, ProXC will exit as well, and the \texttt{START} call in \texttt{main} will return.

\begin{lstlisting}[style={CustomC},caption={Hello World in ProXC}]
void process(void) {
    printf("Hello World!");
}
int main(int argc, char *argv[]) {
    START(process);
}
\end{lstlisting}

A process, which is not the main process, can exit the library with the call \texttt{EXIT}. This will stop the library, and the \texttt{START} call in \texttt{main} will return as if the main process returned. 

\section*{Process Creation}
All process functions must have the function type signature of

\begin{lstlisting}[style={CustomC},frame={},numbers={none}]
void process(void);
\end{lstlisting}

A new process is created with the \texttt{PROC} call, which takes a function pointer to a process function, and zero\hyp{}or\hyp{}more function arguments for the process. Arguments can only be of pointer types. Within the new process, arguments are retrieved with the call \texttt{ARGN}, supplied with the argument number. Arguments are zero\hyp{}indexed. 

\section*{Process Execution}

A new process can be executed synchronously, with the call \texttt{RUN}. This means the spawning process will wait for the new process to finish executing before resuming.

\begin{lstlisting}[style={CustomC},caption={Hello World with multiple processes}]
void printer(void) {
    char *msg = ARGN(0);
    printf("%s", msg);
}
void process(void) {
    RUN( PROC(printer, "Hello ") );
    RUN( PROC(printer, "World!\n") );
}
\end{lstlisting}

There is also asynchronous execution, with the call \texttt{GO}. This means the spawning process will not wait for the new process to finish executing, and will resume execution right after spawning if possible. Listing \ref{lst:async_sync_execution} will alternate between outputting the asynchronous and synchronous message.

\begin{lstlisting}[style={CustomC},caption={Asynchronous and synchronous execution},label={lst:async_sync_execution}]
void never_stop(void) {
    char *msg = ARGN(0);
    for (;;) {
        printf("%s", msg);
        YIELD();
    }
}
void process(void) {
    GO( PROC(never_stop, "I'm asynchronous!\n") );
    RUN( PROC(never_stop, "And I'm synchronous!\n") );
}
\end{lstlisting}

\section*{Yielding}

Note that the process \texttt{never\_stop} in Listing \ref{lst:async_sync_execution} called \texttt{YIELD} in the infinite loop. \texttt{YIELD} is used to give back control to the run\hyp{}time system. ProXC relies on cooperative scheduling, and infinite loops without descheduling points will cause stop the library from regaining control. This is why \texttt{never\_stop} calls \texttt{YIELD}, or only one of the messages would print. 

Yielding can also be used by a process to occasionally give back control if during a data\hyp{}intensive computing over a prolonged time. 

\section*{Composite processes}

It is cumbersome to execute one and one process at the time. Instead, one can define a composite process, which contains multiple processes and their execution order. Defining a sequential execution of processes is used with the call \texttt{SEQ}.  

\begin{lstlisting}[style={CustomC},caption={Sequential composite process execution}]
void process(void) {
    RUN(SEQ(
            PROC(printer, "Hello "),
            PROC(printer, "World!\n")
        )
    );
}
\end{lstlisting}

There is an equivalent call for parallel execution, \texttt{PAR}. Note that the code in Listing \ref{lst:parallel_composite_process_execution} will always print \textit{Chicken} first, every time. This has to do that ProXC is a single\hyp{}core implementation, which means \texttt{PAR} only simulates parallel execution. The order in which processes are defined in a parallel execution will affect which processes are executed first. 

\begin{lstlisting}[style={CustomC},caption={Parallel composite process execution},label={lst:parallel_composite_process_execution}]
void process(void) {
    printf("Which came first? ")
    RUN(PAR(
            PROC(printer, "Chicken\n"),
            PROC(printer, "Egg\n")
        )
    );  
}
\end{lstlisting}

\texttt{SEQ} and \texttt{PAR} can be nested, making intricate and expressive composite trees possible. Listing \ref{lst:nested_composite_process} outputs the sequence of numbers 1, 3, 4, 2, 5, 6.  

\begin{lstlisting}[style={CustomC},caption={Nested composite process},label={lst:nested_composite_process}]
void process(void) {
    RUN(PAR(
            PAR(
                SEQ( PROC(printer, "1\n"), PROC(printer, "2\n") ),
                PROC(printer, "3\n")
            ),
            SEQ(
                PROC(printer, "4\n"),
                PAR( PROC(printer, "5\n"), PROC(printer, "6\n") )
            )
        )
    );  
}
\end{lstlisting}

\section*{Channels}

Independent processes are not very useful for any meaningful computation. Communication between processes is realized through channels. Channels also serve as synchronization points between processes. 

Channels are created with the call \texttt{CHOPEN}, and the data size width of the channel is specified with the call. A channel pointer is returned from the call. This channel pointer can be supplied as arguments when creating composite processes. 

Reading and writing on the channel pointer is done through the calls \texttt{CHREAD} and \texttt{CHWRITE}, respectively. The channel pointer is specified, alongside the data pointer and size.

As the channel is dynamically allocated on the heap, call \texttt{CHCLOSE} to close and cleanup the given channel. Note that this is up to the programmer to do, and is not cared for by the library.

In Listing \ref{lst:chan_comm_sync}, two processes are trying to write a char pointer to the same channel, with one process reading once from the same channel. The two processes is executed asynchronously in parallel, with each assigned to either \textit{chicken} or \textit{egg}. Before writing to the channel, they are randomly sleeping between 1 to 500 milliseconds. When a channel read is successful, either \textit{chicken} or \textit{egg} is printed out. 

\begin{lstlisting}[style={CustomC},caption={Channel communication and synchronization},label={lst:chan_comm_sync}]
static const char * chicken = "chicken", *egg = "egg";
void first(void) {
    Chan *ch = ARGN(0);
    char *entity = ARGN(1);
    int msec = 1 + (rand() % 500); 
    SLEEP(MSEC(msec));
    CHWRITE(ch, &entity, char*);
}
void process(void) {
    Chan *ch = CHOPEN(char*);
    GO(PAR(
            PROC(first, ch, chicken),
            PROC(first, ch, egg)
        )
    );
    char *winner = NULL;
    CHREAD(ch, &winner, char*);
    printf("Winner is %s\n", winner);
    CHCLOSE(ch);
}
\end{lstlisting}

\section*{Process Suspension}

Another API call was used in Listing \ref{lst:chan_comm_sync}, namely \texttt{SLEEP}. This allows a process to be suspended for a given amount of time, without blocking the entire library. If you were to use the normal system call \texttt{sleep}, the entire program would be suspended, as it suspends the kernel\hyp{}thread. 

The suspension time sent to \texttt{SLEEP} has a granularity of microseconds. Three additional convenience macros, \texttt{SEC}, \texttt{MSEC} and \texttt{USEC}, is available to easily convert seconds, milliseconds or microseconds to the granularity of the suspension time. 

\section*{Alternation}

The last construct to introduce is the alternation construct. This allows a process to wait for multiple channel reads at the same time, called events. The events are guarded by a conditional bool, which is only enabled when the condition is true. The alternation construct is by default synchronous, meaning the process alternating waiting for the first available event to be available, and then completing said event. When an event is completed, the alternation construct branches to a corresponding code branch, individual for each event.

The alternation construct is invoked by the call \texttt{ALT}, which take one\hyp{}or\hyp{}more guarded events. For the channel read, the guarded event is created with a \texttt{CHAN\_GUARD}. The alternation construct is used together with a switch block, which represents the individual code block for each event.

\begin{lstlisting}[style={CustomC},caption={Alternation construct setup with channel read guards}]
void process(void) {
    // ...
    switch (ALT(
        // Guard 0
        CHAN_GUARD(x < y, // enabled when x < y equals true
            inCh, &signal, int),
        // Guard 1
        CHAN_GUARD(1, // always enabled
            dataCh, &data, long),
        // Guard 2
        CHAN_GUARD(0, // never enabled
            exitCh, &status, int)
    ) {
    case 0: // Code for Guard 0 
    case 1: // Code for Guard 1 
    case 2: // Code for Guard 2 
    }
}
\end{lstlisting}

The alternation process also supports timeout events, created with a \texttt{TIME\_GUARD}. For a given expiration time, with a granularity of microseconds, the timeout event will become available after expiration. This allows an alternation process to only wait for a given time. 

\begin{lstlisting}[style={CustomC},caption={Alternation construct with timeout}]
void process(void) {
    // ...
    switch (ALT(
        CHAN_GUARD(x < y, 
            inCh, &signal, int),
        TIME_GUARD(1,
            MSEC(500))
    ) {
    case 0: // Code for channel read guard 
    case 1: // Code for timeout guard
    }
}
\end{lstlisting}

Lastly, the alternation process supports skip events, created with a \texttt{SKIP\_GUARD}. This allows the alternation process to select the skip event, if no other events are already available at initialization. 

\begin{lstlisting}[style={CustomC},caption={Alternation construct with skip}]
void process(void) {
    // ...
    switch (ALT(
        CHAN_GUARD(x < y, 
            inCh, &data, int),
        CHAN_GUARD(1, 
            signalCh, &signal, int),
        SKIP_GUARD(1)
    ) {
    case 0: // Code for inCh channel read 
    case 1: // Code for signalCh channel read
    case 2: // Code for skip guard
    }
}
\end{lstlisting}

