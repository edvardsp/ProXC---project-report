% !TEX encoding = UTF-8 Unicode
%!TEX root = main.tex
% !TEX spellcheck = en-US
%%=========================================

\chapter{Discussion}
\label{ch:discussion}

This chapter discusses the practical abilities ProXC achieves, what limitations the current implementation faces, and what could have been implemented if given enough time. Changes in design and implementation which became apparent too late in the project is also taken up for discussion. 

\section{What does ProXC Achieve}

The conceptual driving force of ProXC is to abstract programs into the CSP paradigm. There are mutiple ways to accomplish this, and none of them are the de\hyp{}facto answer. In order to ``\textit{measure}'' how successful ProXC is, is to see how correct and useful the abstraction it introduces is. 

\section{Shortcomings}
\subsection{Fixed stack}
\subsection{Simple scheduler policy}
\subsection{Portability}
\label{sec:portability}



\section{Design and Implementation Improvements}



\section{What is not Implemented}
\subsection{Multi\hyp{}core Support}
\subsection{Replicators}



