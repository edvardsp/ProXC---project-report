% !TEX encoding = UTF-8 Unicode
%!TEX root = main.tex
% !TEX spellcheck = en-US
%%=========================================

\chapter{Introduction}
\label{ch:introduction}

In the last decade one of the leading factors in the ever demanding increase in computer processing power is the shift from singlecore to multicore architectures. This entails a whole new set of challenges and problems, but for programming this means a great potential for speedup if the multicores are utilized. The advent of concurrent programming was not necessarily developed with multicore in mind. But, properly utilizing multi\hyp{}core architectures has the potential of achieving great speedup, and at the same time serving as a great tool to both develop and verify the correctness of concurrent programs. 

The notion of concurrency in programming is not a new thing, and has existed since the 60's. Concurrency regarding computing means multiple computations that are executed in an overlapping fashion, giving the impression that multiple computations are advancing at the same time. Concurrency does not implicitly mean parallelism. Concepts such as semaphores, threads and processes forms its basis on concurrency. 

The introduction of \textit{Communicating Sequential Processes} (CSP) by Tony Hoare \citep{csp} formed a whole new basis within concurrent computing. CSP is on its own a formal language, which describes concurrent systems by defining independent processes, communicating only through message-passing. Not only is CSP a powerful tool to describe concurrent systems, but is also a tool to analyze and check the correctness of concurrent systems. 

CSP has influenced the design of multiple programming languages, such as Occam, XC and Go. Sadly however, many of these programming languages never received the mainstream popularity it deserved, mostly from being hardware locked  or the lack of use of CSP in the industry. The exception here is Go, which has risen in popularity the last few years. This lack of mainstream CSP programming languages, both in use and availability, has resulted in the development of multiple libraries for more popular languages and compilers of existing CSP languages. This includes JCSP \citep{jcsp}, C++CSP2 \citep{c++csp2}, SPoC \citep{spoc}, and KRoC \citep{kroc}. But as this list shows, there does not exist a fully fledged CSP library for C that is both feature rich and mature for real development. 

FIXME

This report details the choice of library, details and implementation for ProXC. Chapter \ref{ch:introduction} briefly explains the current lack of CSP influenced concurrent programming languages and tools, its potential in multicore architectures, and goes on defining the project description. Chapter \ref{ch:background} goes more in depth in related concepts and existing work in CSP modelled concurrency programming. Chapter \ref{ch:proxc_library} explains the choices related to the library, such as why a library and library features. Chapter \ref{ch:design_implementation} goes into detail the design and implementation of the library features. Chapter \ref{ch:examples_usage} presents examples of code, showcasing the different library features in use. Chapter \ref{ch:performance} briefly compares benchmarks of the library compared to other existing CSP languages. Chapter \ref{ch:discussion} discusses what the current status of development, what is missing, and limitations with the library. Chapter \ref{ch:future_work} discusses what is possible for future work on the project. Chapter \ref{ch:conclusion} concludes the report.

