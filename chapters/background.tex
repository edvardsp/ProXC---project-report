% !TEX encoding = UTF-8 Unicode
%!TEX root = main.tex
% !TEX spellcheck = en-US
%%=========================================

\chapter{Background}

In this chapter a more in depth explanation of the different topics introduced in the introduction chapter is presented. These topics cover the required background knowledge for the project, which plays a central role in determining on what kind of development tool is to be implemented, how it is to be implemented, as well as showing what already exists of current solutions and tools.

Each topic is presented on its own, and how it relates and affects the choices of the development tool.

It should be noted that section \ref{sec:csp_formalisms} only includes entries which relates to the choices and development of the end product, even though a great number of entries could fit in that list.

\section{Parallelism vs. Concurrency}

The notion of parallelism and concurrency in programs spawns from the limitations of sequential programs. All programming languages have different ways to express the structure and control flow of a program, but in the end all programs are transformed to machine code which the processor executes \textit{sequentially}. That means for a single processor, only one instruction is executed at a time\footnote{This implies the processor is singlecore}. Concurrency however defines a program into multiple independent sequential programs, which in turn runs each sequential program in interleaving time periods. Even though only one instruction is executed at a time, this gives the impression that multiple sequential programs are advancing at the same time.

Parallelism in this sense refers to multiple programs running on multiple processors in parallel. This might look familiar to what concurrency described, but it is important to not confuse these two terms together. Concurrency only serves as an abstraction for parallelism in a program, allowing the programmer to pretend multiple sequential programs are executed in parallel. This abstraction would be valid for both single and multicore, while parallelism describes a condition which only happens on multicore. A more thorough and complete definition of concurrency and parallelism is described in \citet{benari2006}.

Concurrency and parallelism forms the very core of this project. Section \ref{sec:csp} explains what CSP is, and it forms its basis on concurrency. In section \ref{sec:project_description} the project description  CSP enabling development tool, so understanding the underlying concepts of concurrency is important to understand what CSP entails. Further on, multicore support was also mentioned as a motivation for the potential speedup, as a direct opportunity from the parallel nature in concurrent systems. Parallelism is directly tied to multicore, as both terms describe multiple programs executing simultaneously. 

\section{Threading Models}

\section{Communicating Sequential Programming}
\label{sec:csp}

\section{CSP Influenced Formalisms}
\label{sec:csp_formalisms}

\subsection{Programming Languages}

\subsubsection*{occam}
\subsubsection*{XC}
\subsubsection*{Go}

\subsection{Programming Libraries}

\subsubsection*{C++CSP2}
\subsubsection*{JCSP}

\subsection{Compilers}

\subsubsection*{SPoC}
\subsubsection*{KRoC}


