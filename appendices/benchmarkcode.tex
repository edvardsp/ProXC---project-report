%!TEX root = main.tex
% !TEX spellcheck = en-US
%%=========================================

\chapter{Benchmark Code}
\label{ch:benchmark_code}

\section*{Context Switch Test}

\subsection*{ProXC}

\begin{lstlisting}[caption={Context switch code for ProXC},style={CustomC}]
#include <stdio.h>
#include <stdint.h>
#include <time.h>
#include <proxc.h>
#define X     10
#define ITERS 1000000UL
#define RUNS  1000
void worker(void) {
    for (uint64_t iter = 0; iter < ITERS; ++iter) {
        YIELD();
        //__asm__("");
    }
}
void bench(void) {
    printf("ProXC context bench\n");
    printf("X: %d\n", X);
    printf("ITERS: %lu\n", ITERS);
    printf("RUNS: %d\n", RUNS);
    double summa = 0.0;
    for (int run = 0; run < RUNS; run++) {
        clock_t start, stop;
        start = clock();
        RUN(PAR(
            PROC(worker), 
            PROC(worker), 
            PROC(worker), 
            PROC(worker), 
            PROC(worker), 
            PROC(worker), 
            PROC(worker), 
            PROC(worker), 
            PROC(worker), 
            PROC(worker)
        ));
        stop = clock();
        double elapsed_ns = (double)(stop - start);
        elapsed_ns *= 1000.0 * 1000.0 * 1000.0 / CLOCKS_PER_SEC;
        double loop_time = elapsed_ns / ITERS;
        summa += loop_time;
    }
    printf("average time per loop iter: %fns\n", summa / RUNS);
}
int main(void) {
    proxc_start(bench);
}
\end{lstlisting}

\subsection*{Go}

\begin{lstlisting}[caption={Context switch code for Go},style={CustomGo}]
package main
import (
	"fmt"
	"runtime"
	"sync"
	"time"
)
const (
	X     = 10
	ITERS = 1000000
	RUNS  = 1000
)
func worker(wg *sync.WaitGroup) {
	for i := 0; i < ITERS; i++ {
		runtime.Gosched()
	}
	wg.Done()
}
func main() {
	runtime.GOMAXPROCS(1)
	fmt.Println("Go context bench")
	fmt.Printf("X: %d\n", X)
	fmt.Printf("ITERS: %d\n", ITERS)
	fmt.Printf("RUNS: %d\n", RUNS)
	var summa float64 = 0.0
	var wg sync.WaitGroup
	for run := 0; run < RUNS; run++ {
		wg.Add(X)
		start := time.Now()
		for x := 0; x < X; x++ {
			go worker(&wg)
		}
		wg.Wait()
		elapsed := float64(time.Since(start).Nanoseconds())
		loop_time := elapsed / ITERS
		summa += loop_time
	}
	fmt.Println("")
	fmt.Println("Results:")
	fmt.Printf("average time per loop iter: %fns\n", summa/RUNS)
}
\end{lstlisting}

\subsection*{occam}

\begin{lstlisting}[caption={Context switch code for occam},style={CustomOccam}]
#INCLUDE "hostio.inc"
#USE "hostio.lib"
#H #include <time.h>
PROC worker(VAL INT iters)
  INT iter :
  SEQ iter = 0 FOR iters
    PAR -- hack to force SPoC a context switch
      SKIP
:
PROC bench(CHAN OF SP fs,ts, []INT mem)
  VAL x     IS 1 : 
  VAL iters IS 1000000 :
  VAL runs  IS 1000 :
  INT start, stop, elapsed :
  REAL64 summa, iter :
  SEQ
    #C printf("occam ctx switch\n");
    #C printf("X: %d\n", $x);
    #C printf("ITERS: %d\n", $iters);
    #C printf("RUNS: %d\n", $runs);
    summa := 0.0(REAL64)
    SEQ run = 0 FOR runs
      SEQ
        #C $start = clock();
        PAR i = 0 FOR x
          SEQ iter = 0 FOR iters
            SKIP
        #C $stop = clock();
        #C $elapsed = (double)($stop - $start) * 1000.0 * 1000.0 * 1000.0 / CLOCKS_PER_SEC;
        #C $summa += $elapsed / $iters;
    #C printf("Average: %fns\n", $summa / $runs);
:
\end{lstlisting}

\section*{Commstime Test}

\subsection*{ProXC}

\begin{lstlisting}[caption={Commstime code for ProXC},style={CustomC}]
#include <stdio.h>
#include <stdlib.h>
#include <unistd.h>
#include <time.h>
#include <proxc.h>
#define WARMUP  1000
#define REPEAT  1000000
#define RUNS    1000
void prefix(void) { 
    Chan *a = ARGN(0), *b = ARGN(1);
    int value = 0;
    for(;;) {
        CHWRITE(a, &value, int);
        CHREAD(b, &value, int);
    }
}
void delta(void) { 
    Chan *a = ARGN(0), *c = ARGN(1), *d = ARGN(2);
    int value;
    for (;;) {
        CHREAD(a, &value, int);
        CHWRITE(d, &value, int);
        CHWRITE(c, &value, int);
    }
}
void successor(void) {
    Chan *b = ARGN(0), *c = ARGN(1);
    int value;
    for (;;) {
        CHREAD(c, &value, int);
        value++;
        CHWRITE(b, &value, int);
    }
}
void consumer(void) {
    Chan *d = ARGN(0);
    clock_t start, stop;
    double summa = 0.0, elapsed;
    int x;
    printf("ProXC commstime\n");
    printf("WARMUP: %d\n", WARMUP);
    printf("REPEAT: %d\n", REPEAT);
    printf("RUNS: %d\n", RUNS);
    for (int j = 0; j < RUNS; j++) {
        for (int i = 0; i < WARMUP; i++ ) {
            CHREAD(d, &x, int);
        }
        start = clock();
        for (int i = 0; i < REPEAT; i++ ) {
            CHREAD(d, &x, int);
        }
        stop = clock();
        elapsed = (double)(stop - start);
        elapsed *= 1000.0 * 1000.0 * 1000.0 / CLOCKS_PER_SEC;
        summa += elapsed / REPEAT;
    }
    printf("average: %fns\n", summa / RUNS);
}
void commstime(void) {
    Chan *a = CHOPEN(int);
    Chan *b = CHOPEN(int);
    Chan *c = CHOPEN(int);
    Chan *d = CHOPEN(int);
    GO(PAR(
           PROC(prefix, a, b),
           PROC(delta, a, c, d),
           PROC(successor, b, c)
       )
    );
    RUN( PROC(consumer, d) );
}
int main(void) {
    proxc_start(commstime);
    return 0;
}
\end{lstlisting}

\subsection*{Go}

\begin{lstlisting}[caption={Commstime code for Go},style={CustomGo}]
package main
import (
	"fmt"
	"runtime"
	"time"
)
const (
	WARMUP = 1000
	REPEAT = 1000000
	RUNS   = 100
)
func prefix(a, b chan int) {
	value := 0
	for {
		a <- value
		value = <-b
	}
}
func delta(a, c, d chan int) {
	var value int
	for {
		value = <-a
		d <- value
		c <- value
	}
}
func successor(b, c chan int) {
	var value int
	for {
		value = <-c
		value++
		b <- value
	}
}
func main() {
	runtime.GOMAXPROCS(1)
	fmt.Println("Go commstime")
	fmt.Printf("WARMUP: %d\n", WARMUP)
	fmt.Printf("REPEAT: %d\n", REPEAT)
	fmt.Printf("RUNS: %d\n", RUNS)
	a := make(chan int)
	b := make(chan int)
	c := make(chan int)
	d := make(chan int)
	go prefix(a, b)
	go delta(a, c, d)
	go successor(b, c)
	summa := 0.0
	for run := 0; run < RUNS; run++ {
		for i := 0; i < WARMUP; i++ {
			<-d
		}
		start := time.Now()
		for i := 0; i < REPEAT; i++ {
			<-d
		}
		elapsed := float64(time.Since(start).Nanoseconds())
		summa += elapsed / REPEAT
	}
	fmt.Printf("Average: %fns\n", summa/RUNS)
}
\end{lstlisting}

\subsection*{occam}

\begin{lstlisting}[caption={Commstime code for occam},style={CustomOccam}]
#INCLUDE "hostio.inc"
#USE "hostio.lib"
#H #include <time.h>
PROC prefix(CHAN OF INT a, b)
  INT value :
  SEQ
    value := 0
    WHILE TRUE
      SEQ
        a ! value
        b ? value
:
PROC delta(CHAN OF INT a, c, d)
  INT value :
  WHILE TRUE
    SEQ
      a ? value
      d ! value
      c ! value
:
PROC successor(CHAN OF INT b, c)
  INT value :
  WHILE TRUE
    SEQ
      c ? value
      b ! value + 1 
:
PROC commstime(CHAN OF SP fs,ts, []INT mem)
  CHAN OF INT a, b, c, d :
  VAL warmup IS 1000 :
  VAL repeat IS 1000000 :
  VAL runs   IS 1000 :
  INT start, stop :
  REAL64 summa, elapsed :
  SEQ
    #C printf("occam commstime\n");
    #C printf("WARMUP: %d\n", $warmup);
    #C printf("REPEAT: %d\n", $repeat);
    #C printf("RUNS: %d\n", $runs);
    PAR
      prefix(a, b)
      delta(a, c, d)
      successor(b, c)
      INT x :
      SEQ
        SEQ run = 0 FOR runs
          SEQ
            SEQ i = 0 FOR warmup
              d ? x
            #C $start = clock();
            SEQ i = 0 FOR repeat
              d ? x
            #C $stop = clock();
            #C $elapsed = (double)($stop - $start);
            #C $elapsed *= 1000.0 * 1000.0 * 1000.0 / CLOCKS_PER_SEC; 
            #C $summa += $elapsed / $repeat;
        #C printf("Average: %f\n", $summa / $runs);
        so.exit(fs,ts,sps.success)
:
\end{lstlisting}

\section*{Concurrent Prime Sieve Test}

\subsection*{ProXC}

\begin{lstlisting}[caption={Concurrent prime sieve code for ProXC},style={CustomC}]
#include <stdio.h>
#include <time.h>
#include <proxc.h>
void generate(void) {
    Chan *chlong = ARGN(0);
    long i = 2;
    for (;;) {
        CHWRITE(chlong, &i, long);
        i++;
    }
}
void filter(void) {
    Chan *in_chlong = ARGN(0);
    Chan *out_chlong = ARGN(1);
    long prime = (long)(long *)ARGN(2);
    long i;
    for (;;) {
        CHREAD(in_chlong, &i, long);
        if (i % prime != 0) {
            CHWRITE(out_chlong, &i, long);
        }
    }
}
#define PRIME 4000
static Chan *chs[PRIME+1];
void sieve(void) {
    chs[0] = CHOPEN(long);
    Chan *chlong = chs[0];
    long prime = 0;
    long start, stop;
    double elapsed;
    start = clock();
    GO(PROC(generate, chlong));
    for (long i = 0; i < PRIME; i++) {
        CHREAD(chlong, &prime, long);
        chs[i+1] = CHOPEN(long);
        Chan *new_chlong = chs[i+1];
        GO(PROC(filter, chlong, new_chlong, (void *)prime));
        chlong = new_chlong;
    }
    stop = clock();
    elapsed = (double)(stop - start);
    elapsed *= 1000.0 * 1000.0 / CLOCKS_PER_SEC;
    printf("%.2f\n", elapsed / PRIME);
}
int main(void) {
    proxc_start(sieve);
}
\end{lstlisting}

\subsection*{Go}

\begin{lstlisting}[caption={Concurrent prime sieve code for Go},style={CustomGo}]
package main
import (
	"fmt"
	"runtime"
	"time"
)
func Generate(ch chan<- int) {
	for i := 2; ; i++ {
		ch <- i
	}
}
func Filter(in <-chan int, out chan<- int, prime int) {
	for {
		i := <-in // Receive value from 'in'.
		if i%prime != 0 {
			out <- i // Send 'i' to 'out'.
		}
	}
}
const (
	PRIME = 4000
)
func main() {
	runtime.GOMAXPROCS(1)
	ch := make(chan int)
	var prime int
	start := time.Now()
	go Generate(ch)
	for i := 0; i < PRIME; i++ {
		prime = <-ch
		ch1 := make(chan int)
		go Filter(ch, ch1, prime)
		ch = ch1
	}
	elapsed := float64(time.Since(start).Nanoseconds())
	fmt.Printf("%.2f\n", elapsed/1000.0/PRIME)
}
\end{lstlisting}

\newpage

\subsection*{occam}

\begin{lstlisting}[caption={Concurrent prime sieve code for occam},style={CustomOccam}]
#INCLUDE "hostio.inc"
#USE "hostio.lib"
#H #include <time.h>
PROC generate(CHAN OF INT ch)
  INT i :
  SEQ
    i := 2
    WHILE TRUE
      SEQ
        ch ! i
        i := i + 1
:
PROC filter(CHAN OF INT in, out)
  INT i, prime :
  SEQ
    in ? prime
    WHILE TRUE
      SEQ
        in ? i
        IF
          NOT ((i \ prime) = 0)
            out ! i
          TRUE
            SKIP
:
PROC bench(CHAN OF SP fs,ts, []INT mem)
  VAL numPrime IS 10000 :
  [numPrime]CHAN OF INT chs :
  INT prime :
  INT start, stop :
  REAL64 elapsed :
  SEQ
    #C $start = clock();
    PAR
      generate(chs[0])
      PAR i = 0 FOR numPrime-1
        filter(chs[i], chs[i+1])
      SEQ
        chs[numPrime-1] ? prime
        #C $stop = clock();
        #C $elapsed = (double)($stop - $start) * 1000.0 * 1000.0 / CLOCKS_PER_SEC;
        #C printf("%d: %d\n", $numPrime, $prime);
        #C printf("%.2f\n", $elapsed / $numPrime);
        so.exit(fs,ts,sps.success)
:

\end{lstlisting}

